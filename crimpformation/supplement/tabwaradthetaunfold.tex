%\documentclass{article}
%\usepackage{lscape}
%\begin{document}

\begin{table}[htp]
\centering
\caption{Measured amplitude and wavelength, by the SF technique, converted to values of intrinsic radius ($a$) and angle between unfoldings ($\theta$) for 8 sheep with unfolded helix crimp form. Comparison with OFDA mean curvature ($C$) converted to radius of curvature ($R$).}
\label{tab:waradthetaunfold}
\vspace{0.1in}
\begin{tabular}{|p{0.4in}|p{0.4in}|p{0.4in}|p{0.6in}|p{0.6in}|p{0.5in}|p{0.6in}|} \hline
  Sheep No &   Wave length  & Ampl itude & Intrinsic radius (mm)  & Angle between unfoldings (degrees) &  OFDA mean curvature (rads/mm) & OFDA radius of curvature (mm) \\  \hline
  14355 & 2.32  & 0.358  & 0.648 & 126.7  & 1.349 & 0.7412   \\ 
  14406 & 2.52  & 0.406  & 0.692 & 131.1 & 1.178 & 0.8488   \\
  14408 & 2.62  & 0.460  & 0.696 & 140.3 & 1.309 & 0.7639   \\
  14413 & 2.84  & 0.483  & 0.763 & 136.9 & 1.026 & 0.9744   \\ 
  14525 & 2.94  & 0.500  & 0.790 & 136.9 & 1.253 & 0.7980   \\
  14526 & 2.54  & 0.355  & 0.745 & 116.8 & 1.278 & 0.7827   \\
  14586 & 2.39  & 0.441  & 0.625 & 145.7 & 1.321 & 0.7569   \\ 
  14593 & 2.48  & 0.372  & 0.702 & 123.8 & 1.119 & 0.8938   \\ \hline
\end{tabular}
\end{table}

%\end{document}
