%\documentclass{article}
%\usepackage{lscape}
%\begin{document}

\begin{center}
\begin{landscape}
\begin{table}[h]
\caption{Approximate measurements for sheep representing Ryder's 4 stages of Merino evolution, extended to the modern SRS Merino. Sources lines 1 to 3 Ryder(1992)~\cite{ryde:92}, line 4 Carter(1968)~\cite{cart:68}, line 5 Watts(2017)~\cite{watt:17}. Correlated changes in birthcoat are not evidence based, but are inferred from our present understanding of birthcoats. Base is the primary central fibre before prenatal check, Post-check is the primary fibres after prenatal check, ie the 'strength' of the check} 
\label{tab:evolstages}
\vspace{0.1in}
\begin{tabular}{|p{0.6in}|p{0.6in}|p{0.4in}|p{0.4in}|p{0.6in}|p{0.7in}|p{0.7in}|p{0.7in}|p{0.7in}|}  \hline
 
  \multicolumn{2}{|c|}{Evolution} & \multicolumn{5}{c|}{Adult Fleece} & \multicolumn{2}{c|}{Birthcoat Fibres} \\ \hline
  Stage &  Chronology &  Dp & Ds & S/P ratio & Medullation & Shedding  & Base & Post-check \\ \hline
  Wild   & Neolithic (10000-3000BC) & 80-200 & 14-16 & 3-5 & Outercoat kemp & Undercoat sheds & Coarse sickle tip & Very Coarse fibre \\ \hline
  Hairy Medium Wool & Bronze Age (3000 - 1000BC) & 40-140 & 18-28 & 4-5 & Outercoat hair & Undercoat sheds & Coarse sickle tip & Medium fibre \\ \hline
  Generalised Medium Wool & Iron Age (1000BC - 700 AD) & 20-50 & 18-35 & 5-7  & No outercoat? & Secondaries continuous & Coarse sickle tip & Fine fibre \\  \hline
  Finewool Merino & (1200AD - present) & 18-20 & 17-19 & 15-25 & No outercoat & Secondaries continuous & Medium sickle tip & Fine fibre\\  \hline
  SRS Fine Merino & (1990AD - present) & 14-17 & 17-19 & 25-40 & No outercoat & Secondaries continuous  & Fine long sickle tip  & Fine fibres \\ \hline
\end{tabular}
\end{table}
\end{landscape}
\end{center}

%\end{document}
